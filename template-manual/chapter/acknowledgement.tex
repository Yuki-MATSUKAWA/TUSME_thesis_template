%%%%%%%%%%%%%%%%
%%%%% 謝辞 %%%%%
%%%%%%%%%%%%%%%%
\acknowledge

謝辞は非常に重要な章です.
学位論文執筆にあたりお世話になった人は必ず書きましょう.
私は基本的に謝辞は自由に書いてほしいと思っていますが,明らかな間違いを書いてくる人を結構見てきました.
これまで添削してきた論文の中でよく見た間違いや書き方の例などをここでまとめておきます.

まずは全体に共通する注意事項です.
\begin{itemize}
    \item 論文本体は基本的に常体(だ・である)で書きますが,謝辞に限り敬体(です・ます)で書いても構いません.
    \item 氏名の書き間違い等には十分気をつけてください.
    間違えると大変失礼です.
    心配な場合は大学や研究室のウェブページ等からコピーアンドペーストするとよいでしょう.
    特殊な文字が氏名に入っている場合の出力方法は第~\ref{sec:unicode}~節を参照してください.
    \item また,所属名の間違いにも気をつけましょう.
    卒業論文執筆時と修士論文執筆時で所属名が変わっていることもあります(大学の学部再編等の事情による).
    \item 基本的な日本の大学教員の職階(役職名)は上から順に以下の通りです.
    \begin{enumerate}
        \item 教授
        \item 准教授
        \item 講師
        \item 助教
        \item 助手
    \end{enumerate}
    大学によっては上記職階の一部が設置されていない場合があります.
    また,よくある間違いとして,准教授や助教を「助教授」と書くことが挙げられます.
    助教授はかつて実際に存在した職階ですが,2007 年の学校教育法の改正に伴い,准教授へと名称が変更されました.
    また,助教が助教授の略称だと勘違いしているケースもよく見ます.
    \item 普段大学で○○教授と呼ぶことはありませんが,学位論文の謝辞では「フルネーム$+$役職名」で記載しましょう.
    ○○教授や○○准教授と比較して○○講師や○○助教という言い方はあまり聞きませんが,統一して役職名で書いてください.
    その際,役職名を間違えると大変失礼です.
    卒業論文執筆時に准教授だった先生が修士論文執筆時には教授に昇進していることもあります.
    卒論の謝辞を使いまわすのではなく,よく確認しておくように.
    \item 修士課程や博士後期課程における「課程」を「過程」と書く間違いをよく見るので気をつけましょう.
    \item また,大学の場合は「卒業」で大学院の場合は「修了」です.
    「修了」を「終了」と書いてしまうミスもたまに見ます.
    \item 謝辞はどうしても同じような文言や文末表現が連続しがちです.
    しかし,感謝を伝えるためにも表現を少しずつ変えてください.
\end{itemize}

次に,おおまかな書く順番とそれぞれの注意事項は下記の通りです.
必ずこの順番を守らなければいけないというわけではありませんが,このような順であることが多いです.
\begin{enumerate}
    \item 指導教員(主査)
    \begin{itemize}
        \item 書き方の例:指導教員である東京理科大学創域理工学部機械航空宇宙工学科の○○教授は……
        \item 大学によっては教員の所属が大学院の場合もありますが,東京理科大学の場合は学部です.
        \item よくある間違いとして「指導教官の……」と書くものです.教官という言い回しは昔の国立大学の教員のものです.
        正しくは「指導教員」です.
        \item 稀に実質的な指導教員と主査が異なる場合があります.
    \end{itemize}
    \item 共同研究者
    \begin{itemize}
        \item 書き方の例:例:○○研究所の○○博士は……
        \item 共同研究者は大学教員の場合もあれば大学以外の研究機関の職員の場合もあります.所属名と役職名をよく確認しておきましょう.
    \end{itemize}
    \item 副査の先生方
    \begin{itemize}
        \item 書き方の例:○○教授と○○准教授には本論文の副査を引き受けていただき……
        \item 卒業論文の場合は副査がありませんが,修士論文の場合は 2 名の副査があります.
    \end{itemize}
    \item その他お世話になった学生以外の先生方・研究者
    \begin{itemize}
        \item 書き方の例:本研究室所属の○○博士研究員は……
        \item 研究室に博士研究員(いわゆるポスドク)や直接の自分の指導教員ではない助教や秘書の方がいる場合も忘れずに書いておきましょう.
    \end{itemize}
    \item 研究助成元や計算機センター等
    \begin{itemize}
        \item 書き方の例:直接数値計算の一部は東北大学サイバーサイエンスセンター大規模計算システム AOBA を利用しました.
        \item 研究遂行にあたり助成等を受けた場合に記載します.奨学金をもらっている場合もここに書きましょう.
    \end{itemize}
    \item 研究室メンバー
    \begin{itemize}
        \item 書き方の例:博士後期課程 1 年の○○氏は……
        \item 研究室メンバー全員を列挙しなければいけないわけではありません.論文執筆にあたって本当に貢献した人を書いてください.全員の貢献があっての学位論文だと思うのであれば全員書いてもいいと思います.
        \item 並べ方は基本的に貢献度順です.自分と同じ研究班の人をまずは優先的に書きましょう.
        \item 貢献度が同じだと判断した場合は上からの学年順に並べましょう.
        \item 貢献度と学年が同じ人は五十音順に並べましょう.
        \item 日本学術振興会特別研究員に採用されている博士後期課程学生がいる場合は肩書きとして併記することが多いです.
    \end{itemize}
    \item 家族
    \begin{itemize}
        \item 書き方の例:最後に,私の大学院進学に対して理解を示し,常に私を気にかけていただいた祖母と両親に感謝申し上げます.
        \item 自分を応援してくれた家族への感謝も忘れずに.
        \item 残念ながら自分の卒業・修了を見届けられずにご家族が亡くなってしまう場合もあります.その際は「天国で見守ってくれている○○」などと書くことが多いです.
        \item なお,家庭環境等の事情により家族を記載したくない場合はこの限りではありません.心の中で感謝しておきましょう.
    \end{itemize}
\end{enumerate}

