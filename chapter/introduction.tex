%%%%%%%%%%%%%%%%
%%%%% 序論 %%%%%
%%%%%%%%%%%%%%%%

\chapter{序論}
\label{ch:introduction}

% ダミーテキスト
\jalipsum[1-2]{wagahai}

\section{研究背景}
\label{sec:background}

% ダミーテキスト
\jalipsum[3-4]{wagahai}

\section{先行研究}
\label{sec:previous}

\subsection{A の先行研究}
\label{ssec:previous_A}

\subsection{B の先行研究}
\label{ssec:previous_B}


\section{本研究の意義・目的}
\label{sec:objective}

\begin{equation}
    \pdv{u_r}{t} + (\mathbf{u}\cdot\nabla)u_r - \frac{u_\theta^2}{r} = 
    -\frac{1}{\rho}\pdv{p}{r} + \nu\ab(\nabla^2 u_r - \frac{u_r}{r^2} - \frac{2}{r^2}\pdv{u_\theta}{\theta})
    \label{eq:NSr2}
\end{equation}

\begin{figure}[tp]
    \centering
    \includegraphics[width=0.5\columnwidth]{figure/ppt-sample-crop.pdf}
    \caption{1枚の図.}
    \label{fig:one_figure}
\end{figure}

\subsection{参考文献の練習}
\label{ssec:bibtex}

\citet{Matsukawa:PoF2022}

TCPFの乱流遷移現象は複雑である\citep{Matsukawa:PoF2022}.

FSC相似解に関する文献は\citet{Liu:2021}があります.

境界条件の扱いに関しては\citet{Guastoni:2021}を参考にしました.

\subsection{単位の練習}
\label{ssec:unit}

$\si[per-mode=symbol]{\watt\per{\square\meter\kelvin}}$

この文は後で追加した文です.