%%%%%%%%%%%%%%%%%%%%%%%%%%%%%%%%%%%%%%%%%%%%%%%%%%%%%%%%%%%%%%%%%%%%%%%
%%%
%%%         東京理科大学大学院 創域理工学研究科 機械航空宇宙工学専攻
%%%                   【非公式】修士論文要旨 テンプレート
%%%
%%%      <https://github.com/tsukahara-lab/TUS-ME_thesis_template>
%%%
%%%                                  v1.0.0 Yuki MATSUKAWA 15 Dec. 2021
%%%                                  v2.0.3 Yuki MATSUKAWA 27 Dec. 2023
%%%
%%%%%%%%%%%%%%%%%%%%%%%%%%%%%%%%%%%%%%%%%%%%%%%%%%%%%%%%%%%%%%%%%%%%%%%

%%% 文書クラスの設定 %%%
\documentclass[
    paper=a4paper,      % A4 用紙サイズ
    article,            % article 相当の文書クラス
    fleqn,              % 数式を左寄せ
    fontsize=12pt,      % 欧文サイズ 12 pt
    jafontsize=12pt,    % 和文サイズ 12 pt
    head_space=30mm,    % 天の余白
    foot_space=20mm,    % 地の余白
    gutter=18mm,        % のどの余白
    fore-edge=18mm      % 小口の余白
    ]{jlreq}            % jlreq クラスを使用

%%% abstract style %%%
% 修論要旨設定ファイル
\usepackage{settings_master}

% 行番号の表示
% 添削時には行番号を付けるとわかりやすい
% 提出時にはコメントアウトする
\linenumbers

\begin{document}

% 1 行あたり文字数の指定
\mojiparline{35}
% 1 ページあたり行数の指定
\linesparpage{30}

\begin{center}
% 修士論文題目
ここには修士論文のタイトルを入れます.\\ 一文字でも間違えたら受理されません.
% 修士論文題目
\end{center}

\vskip\baselineskip
\noindent
% 姓と名の間に「全角」スペース忘れずに
\begin{flushright}
    \begin{tabular}{r@{\hspace{3\zw}}r@{\hspace{0pt}\vspace{9pt}}}
        機械航空宇宙工学専攻 &
        % 自分の氏名を記入
        % \\ は消さない
        姓姓 名名 \\
        指導教員 &
        % 指導教員の氏名を記入
        姓姓 名名
    \end{tabular}
\end{flushright}

\vskip\baselineskip
%%% ここから書き始める %%%
アブストラクトアブストラクトアブストラクトアブストラクトアブストラクトアブストラクトアブストラクトアブストラクトアブストラクト
アブストラクトアブストラクトアブストラクトアブストラクトアブストラクトアブストラクトアブストラクトアブストラクトアブストラクト
アブストラクトアブストラクトアブストラクトアブストラクトアブストラクトアブストラクトアブストラクトアブストラクトアブストラクト
アブストラクトアブストラクトアブストラクトアブストラクトアブストラクトアブストラクトアブストラクトアブストラクトアブストラクト
アブストラクトアブストラクトアブストラクトアブストラクトアブストラクトアブストラクトアブストラクトアブストラクトアブストラクト
アブストラクトアブストラクトアブストラクトアブストラクトアブストラクトアブストラクトアブストラクトアブストラクトアブストラクト
アブストラクトアブストラクトアブストラクトアブストラクトアブストラクトアブストラクトアブストラクトアブストラクトアブストラクト.

アブストラクトアブストラクトアブストラクトアブストラクトアブストラクトアブストラクトアブストラクトアブストラクトアブストラクト
アブストラクトアブストラクトアブストラクトアブストラクトアブストラクトアブストラクトアブストラクトアブストラクトアブストラクト
アブストラクトアブストラクトアブストラクトアブストラクトアブストラクトアブストラクトアブストラクトアブストラクトアブストラクト
アブストラクトアブストラクトアブストラクトアブストラクトアブストラクトアブストラクトアブストラクトアブストラクトアブストラクト
アブストラクトアブストラクトアブストラクトアブストラクトアブストラクトアブストラクトアブストラクトアブストラクトアブストラクト
アブストラクトアブストラクトアブストラクトアブストラクトアブストラクトアブストラクトアブストラクトアブストラクトアブストラクト
アブストラクトアブストラクトアブストラクトアブストラクトアブストラクトアブストラクトアブストラクトアブストラクトアブストラクト.

図~\ref{fig:abst1}は虎,図~\ref{fig:abst2}も虎.
アブストラクトアブストラクトアブストラクトアブストラクトアブストラクトアブストラクトアブストラクトアブストラクトアブストラクト
アブストラクトアブストラクトアブストラクトアブストラクトアブストラクトアブストラクトアブストラクトアブストラクトアブストラクト
アブストラクトアブストラクトアブストラクトアブストラクトアブストラクトアブストラクトアブストラクトアブストラクトアブストラクト
アブストラクトアブストラクトアブストラクトアブストラクトアブストラクトアブストラクトアブストラクトアブストラクトアブストラクト
アブストラクトアブストラクトアブストラクトアブストラクトアブストラクトアブストラクトアブストラクトアブストラクトアブストラクト
アブストラクトアブストラクトアブストラクトアブストラクトアブストラクトアブストラクトアブストラクトアブストラクトアブストラクト
アブストラクトアブストラクトアブストラクトアブストラクトアブストラクトアブストラクトアブストラクトアブストラクトアブストラクト
アブストラクトアブストラクトアブストラクトアブストラクトアブストラクトアブストラクトアブストラクトアブストラクトアブストラクト
アブストラクトアブストラクトアブストラクトアブストラクトアブストラクトアブストラクト.

\begin{figure}[b]
	\centering
	\begin{minipage}{0.45\columnwidth}
		\centering
		\includegraphics[width=\columnwidth]{example-image-a}
		\caption{Example 1.}
		\label{fig:abst1}
	\end{minipage}
	\hspace{10mm}
	\begin{minipage}{0.45\columnwidth}
		\centering
		\includegraphics[width=\columnwidth]{example-image-b}
		\caption{Example 2.}
		\label{fig:abst2}
	\end{minipage}
\end{figure}

%%% ここまで %%%

\end{document}
